
% Default to the notebook output style

    


% Inherit from the specified cell style.




    
\documentclass{article}

    
    
    \usepackage{graphicx} % Used to insert images
    \usepackage{adjustbox} % Used to constrain images to a maximum size 
    \usepackage{color} % Allow colors to be defined
    \usepackage{enumerate} % Needed for markdown enumerations to work
    \usepackage{geometry} % Used to adjust the document margins
    \usepackage{amsmath} % Equations
    \usepackage{amssymb} % Equations
    \usepackage[mathletters]{ucs} % Extended unicode (utf-8) support
    \usepackage[utf8x]{inputenc} % Allow utf-8 characters in the tex document
    \usepackage{fancyvrb} % verbatim replacement that allows latex
    \usepackage{grffile} % extends the file name processing of package graphics 
                         % to support a larger range 
    % The hyperref package gives us a pdf with properly built
    % internal navigation ('pdf bookmarks' for the table of contents,
    % internal cross-reference links, web links for URLs, etc.)
    \usepackage{hyperref}
    \usepackage{longtable} % longtable support required by pandoc >1.10
    \usepackage{booktabs}  % table support for pandoc > 1.12.2
    

    
    
    \definecolor{orange}{cmyk}{0,0.4,0.8,0.2}
    \definecolor{darkorange}{rgb}{.71,0.21,0.01}
    \definecolor{darkgreen}{rgb}{.12,.54,.11}
    \definecolor{myteal}{rgb}{.26, .44, .56}
    \definecolor{gray}{gray}{0.45}
    \definecolor{lightgray}{gray}{.95}
    \definecolor{mediumgray}{gray}{.8}
    \definecolor{inputbackground}{rgb}{.95, .95, .85}
    \definecolor{outputbackground}{rgb}{.95, .95, .95}
    \definecolor{traceback}{rgb}{1, .95, .95}
    % ansi colors
    \definecolor{red}{rgb}{.6,0,0}
    \definecolor{green}{rgb}{0,.65,0}
    \definecolor{brown}{rgb}{0.6,0.6,0}
    \definecolor{blue}{rgb}{0,.145,.698}
    \definecolor{purple}{rgb}{.698,.145,.698}
    \definecolor{cyan}{rgb}{0,.698,.698}
    \definecolor{lightgray}{gray}{0.5}
    
    % bright ansi colors
    \definecolor{darkgray}{gray}{0.25}
    \definecolor{lightred}{rgb}{1.0,0.39,0.28}
    \definecolor{lightgreen}{rgb}{0.48,0.99,0.0}
    \definecolor{lightblue}{rgb}{0.53,0.81,0.92}
    \definecolor{lightpurple}{rgb}{0.87,0.63,0.87}
    \definecolor{lightcyan}{rgb}{0.5,1.0,0.83}
    
    % commands and environments needed by pandoc snippets
    % extracted from the output of `pandoc -s`
    \DefineVerbatimEnvironment{Highlighting}{Verbatim}{commandchars=\\\{\}}
    % Add ',fontsize=\small' for more characters per line
    \newenvironment{Shaded}{}{}
    \newcommand{\KeywordTok}[1]{\textcolor[rgb]{0.00,0.44,0.13}{\textbf{{#1}}}}
    \newcommand{\DataTypeTok}[1]{\textcolor[rgb]{0.56,0.13,0.00}{{#1}}}
    \newcommand{\DecValTok}[1]{\textcolor[rgb]{0.25,0.63,0.44}{{#1}}}
    \newcommand{\BaseNTok}[1]{\textcolor[rgb]{0.25,0.63,0.44}{{#1}}}
    \newcommand{\FloatTok}[1]{\textcolor[rgb]{0.25,0.63,0.44}{{#1}}}
    \newcommand{\CharTok}[1]{\textcolor[rgb]{0.25,0.44,0.63}{{#1}}}
    \newcommand{\StringTok}[1]{\textcolor[rgb]{0.25,0.44,0.63}{{#1}}}
    \newcommand{\CommentTok}[1]{\textcolor[rgb]{0.38,0.63,0.69}{\textit{{#1}}}}
    \newcommand{\OtherTok}[1]{\textcolor[rgb]{0.00,0.44,0.13}{{#1}}}
    \newcommand{\AlertTok}[1]{\textcolor[rgb]{1.00,0.00,0.00}{\textbf{{#1}}}}
    \newcommand{\FunctionTok}[1]{\textcolor[rgb]{0.02,0.16,0.49}{{#1}}}
    \newcommand{\RegionMarkerTok}[1]{{#1}}
    \newcommand{\ErrorTok}[1]{\textcolor[rgb]{1.00,0.00,0.00}{\textbf{{#1}}}}
    \newcommand{\NormalTok}[1]{{#1}}
    
    % Define a nice break command that doesn't care if a line doesn't already
    % exist.
    \def\br{\hspace*{\fill} \\* }
    % Math Jax compatability definitions
    \def\gt{>}
    \def\lt{<}
    % Document parameters
    \title{DATA MINING PROJECT}
    
    
    

    % Pygments definitions
    
\makeatletter
\def\PY@reset{\let\PY@it=\relax \let\PY@bf=\relax%
    \let\PY@ul=\relax \let\PY@tc=\relax%
    \let\PY@bc=\relax \let\PY@ff=\relax}
\def\PY@tok#1{\csname PY@tok@#1\endcsname}
\def\PY@toks#1+{\ifx\relax#1\empty\else%
    \PY@tok{#1}\expandafter\PY@toks\fi}
\def\PY@do#1{\PY@bc{\PY@tc{\PY@ul{%
    \PY@it{\PY@bf{\PY@ff{#1}}}}}}}
\def\PY#1#2{\PY@reset\PY@toks#1+\relax+\PY@do{#2}}

\expandafter\def\csname PY@tok@gd\endcsname{\def\PY@tc##1{\textcolor[rgb]{0.63,0.00,0.00}{##1}}}
\expandafter\def\csname PY@tok@gu\endcsname{\let\PY@bf=\textbf\def\PY@tc##1{\textcolor[rgb]{0.50,0.00,0.50}{##1}}}
\expandafter\def\csname PY@tok@gt\endcsname{\def\PY@tc##1{\textcolor[rgb]{0.00,0.27,0.87}{##1}}}
\expandafter\def\csname PY@tok@gs\endcsname{\let\PY@bf=\textbf}
\expandafter\def\csname PY@tok@gr\endcsname{\def\PY@tc##1{\textcolor[rgb]{1.00,0.00,0.00}{##1}}}
\expandafter\def\csname PY@tok@cm\endcsname{\let\PY@it=\textit\def\PY@tc##1{\textcolor[rgb]{0.25,0.50,0.50}{##1}}}
\expandafter\def\csname PY@tok@vg\endcsname{\def\PY@tc##1{\textcolor[rgb]{0.10,0.09,0.49}{##1}}}
\expandafter\def\csname PY@tok@m\endcsname{\def\PY@tc##1{\textcolor[rgb]{0.40,0.40,0.40}{##1}}}
\expandafter\def\csname PY@tok@mh\endcsname{\def\PY@tc##1{\textcolor[rgb]{0.40,0.40,0.40}{##1}}}
\expandafter\def\csname PY@tok@go\endcsname{\def\PY@tc##1{\textcolor[rgb]{0.53,0.53,0.53}{##1}}}
\expandafter\def\csname PY@tok@ge\endcsname{\let\PY@it=\textit}
\expandafter\def\csname PY@tok@vc\endcsname{\def\PY@tc##1{\textcolor[rgb]{0.10,0.09,0.49}{##1}}}
\expandafter\def\csname PY@tok@il\endcsname{\def\PY@tc##1{\textcolor[rgb]{0.40,0.40,0.40}{##1}}}
\expandafter\def\csname PY@tok@cs\endcsname{\let\PY@it=\textit\def\PY@tc##1{\textcolor[rgb]{0.25,0.50,0.50}{##1}}}
\expandafter\def\csname PY@tok@cp\endcsname{\def\PY@tc##1{\textcolor[rgb]{0.74,0.48,0.00}{##1}}}
\expandafter\def\csname PY@tok@gi\endcsname{\def\PY@tc##1{\textcolor[rgb]{0.00,0.63,0.00}{##1}}}
\expandafter\def\csname PY@tok@gh\endcsname{\let\PY@bf=\textbf\def\PY@tc##1{\textcolor[rgb]{0.00,0.00,0.50}{##1}}}
\expandafter\def\csname PY@tok@ni\endcsname{\let\PY@bf=\textbf\def\PY@tc##1{\textcolor[rgb]{0.60,0.60,0.60}{##1}}}
\expandafter\def\csname PY@tok@nl\endcsname{\def\PY@tc##1{\textcolor[rgb]{0.63,0.63,0.00}{##1}}}
\expandafter\def\csname PY@tok@nn\endcsname{\let\PY@bf=\textbf\def\PY@tc##1{\textcolor[rgb]{0.00,0.00,1.00}{##1}}}
\expandafter\def\csname PY@tok@no\endcsname{\def\PY@tc##1{\textcolor[rgb]{0.53,0.00,0.00}{##1}}}
\expandafter\def\csname PY@tok@na\endcsname{\def\PY@tc##1{\textcolor[rgb]{0.49,0.56,0.16}{##1}}}
\expandafter\def\csname PY@tok@nb\endcsname{\def\PY@tc##1{\textcolor[rgb]{0.00,0.50,0.00}{##1}}}
\expandafter\def\csname PY@tok@nc\endcsname{\let\PY@bf=\textbf\def\PY@tc##1{\textcolor[rgb]{0.00,0.00,1.00}{##1}}}
\expandafter\def\csname PY@tok@nd\endcsname{\def\PY@tc##1{\textcolor[rgb]{0.67,0.13,1.00}{##1}}}
\expandafter\def\csname PY@tok@ne\endcsname{\let\PY@bf=\textbf\def\PY@tc##1{\textcolor[rgb]{0.82,0.25,0.23}{##1}}}
\expandafter\def\csname PY@tok@nf\endcsname{\def\PY@tc##1{\textcolor[rgb]{0.00,0.00,1.00}{##1}}}
\expandafter\def\csname PY@tok@si\endcsname{\let\PY@bf=\textbf\def\PY@tc##1{\textcolor[rgb]{0.73,0.40,0.53}{##1}}}
\expandafter\def\csname PY@tok@s2\endcsname{\def\PY@tc##1{\textcolor[rgb]{0.73,0.13,0.13}{##1}}}
\expandafter\def\csname PY@tok@vi\endcsname{\def\PY@tc##1{\textcolor[rgb]{0.10,0.09,0.49}{##1}}}
\expandafter\def\csname PY@tok@nt\endcsname{\let\PY@bf=\textbf\def\PY@tc##1{\textcolor[rgb]{0.00,0.50,0.00}{##1}}}
\expandafter\def\csname PY@tok@nv\endcsname{\def\PY@tc##1{\textcolor[rgb]{0.10,0.09,0.49}{##1}}}
\expandafter\def\csname PY@tok@s1\endcsname{\def\PY@tc##1{\textcolor[rgb]{0.73,0.13,0.13}{##1}}}
\expandafter\def\csname PY@tok@sh\endcsname{\def\PY@tc##1{\textcolor[rgb]{0.73,0.13,0.13}{##1}}}
\expandafter\def\csname PY@tok@sc\endcsname{\def\PY@tc##1{\textcolor[rgb]{0.73,0.13,0.13}{##1}}}
\expandafter\def\csname PY@tok@sx\endcsname{\def\PY@tc##1{\textcolor[rgb]{0.00,0.50,0.00}{##1}}}
\expandafter\def\csname PY@tok@bp\endcsname{\def\PY@tc##1{\textcolor[rgb]{0.00,0.50,0.00}{##1}}}
\expandafter\def\csname PY@tok@c1\endcsname{\let\PY@it=\textit\def\PY@tc##1{\textcolor[rgb]{0.25,0.50,0.50}{##1}}}
\expandafter\def\csname PY@tok@kc\endcsname{\let\PY@bf=\textbf\def\PY@tc##1{\textcolor[rgb]{0.00,0.50,0.00}{##1}}}
\expandafter\def\csname PY@tok@c\endcsname{\let\PY@it=\textit\def\PY@tc##1{\textcolor[rgb]{0.25,0.50,0.50}{##1}}}
\expandafter\def\csname PY@tok@mf\endcsname{\def\PY@tc##1{\textcolor[rgb]{0.40,0.40,0.40}{##1}}}
\expandafter\def\csname PY@tok@err\endcsname{\def\PY@bc##1{\setlength{\fboxsep}{0pt}\fcolorbox[rgb]{1.00,0.00,0.00}{1,1,1}{\strut ##1}}}
\expandafter\def\csname PY@tok@kd\endcsname{\let\PY@bf=\textbf\def\PY@tc##1{\textcolor[rgb]{0.00,0.50,0.00}{##1}}}
\expandafter\def\csname PY@tok@ss\endcsname{\def\PY@tc##1{\textcolor[rgb]{0.10,0.09,0.49}{##1}}}
\expandafter\def\csname PY@tok@sr\endcsname{\def\PY@tc##1{\textcolor[rgb]{0.73,0.40,0.53}{##1}}}
\expandafter\def\csname PY@tok@mo\endcsname{\def\PY@tc##1{\textcolor[rgb]{0.40,0.40,0.40}{##1}}}
\expandafter\def\csname PY@tok@kn\endcsname{\let\PY@bf=\textbf\def\PY@tc##1{\textcolor[rgb]{0.00,0.50,0.00}{##1}}}
\expandafter\def\csname PY@tok@mi\endcsname{\def\PY@tc##1{\textcolor[rgb]{0.40,0.40,0.40}{##1}}}
\expandafter\def\csname PY@tok@gp\endcsname{\let\PY@bf=\textbf\def\PY@tc##1{\textcolor[rgb]{0.00,0.00,0.50}{##1}}}
\expandafter\def\csname PY@tok@o\endcsname{\def\PY@tc##1{\textcolor[rgb]{0.40,0.40,0.40}{##1}}}
\expandafter\def\csname PY@tok@kr\endcsname{\let\PY@bf=\textbf\def\PY@tc##1{\textcolor[rgb]{0.00,0.50,0.00}{##1}}}
\expandafter\def\csname PY@tok@s\endcsname{\def\PY@tc##1{\textcolor[rgb]{0.73,0.13,0.13}{##1}}}
\expandafter\def\csname PY@tok@kp\endcsname{\def\PY@tc##1{\textcolor[rgb]{0.00,0.50,0.00}{##1}}}
\expandafter\def\csname PY@tok@w\endcsname{\def\PY@tc##1{\textcolor[rgb]{0.73,0.73,0.73}{##1}}}
\expandafter\def\csname PY@tok@kt\endcsname{\def\PY@tc##1{\textcolor[rgb]{0.69,0.00,0.25}{##1}}}
\expandafter\def\csname PY@tok@ow\endcsname{\let\PY@bf=\textbf\def\PY@tc##1{\textcolor[rgb]{0.67,0.13,1.00}{##1}}}
\expandafter\def\csname PY@tok@sb\endcsname{\def\PY@tc##1{\textcolor[rgb]{0.73,0.13,0.13}{##1}}}
\expandafter\def\csname PY@tok@k\endcsname{\let\PY@bf=\textbf\def\PY@tc##1{\textcolor[rgb]{0.00,0.50,0.00}{##1}}}
\expandafter\def\csname PY@tok@se\endcsname{\let\PY@bf=\textbf\def\PY@tc##1{\textcolor[rgb]{0.73,0.40,0.13}{##1}}}
\expandafter\def\csname PY@tok@sd\endcsname{\let\PY@it=\textit\def\PY@tc##1{\textcolor[rgb]{0.73,0.13,0.13}{##1}}}

\def\PYZbs{\char`\\}
\def\PYZus{\char`\_}
\def\PYZob{\char`\{}
\def\PYZcb{\char`\}}
\def\PYZca{\char`\^}
\def\PYZam{\char`\&}
\def\PYZlt{\char`\<}
\def\PYZgt{\char`\>}
\def\PYZsh{\char`\#}
\def\PYZpc{\char`\%}
\def\PYZdl{\char`\$}
\def\PYZhy{\char`\-}
\def\PYZsq{\char`\'}
\def\PYZdq{\char`\"}
\def\PYZti{\char`\~}
% for compatibility with earlier versions
\def\PYZat{@}
\def\PYZlb{[}
\def\PYZrb{]}
\makeatother


    % Exact colors from NB
    \definecolor{incolor}{rgb}{0.0, 0.0, 0.5}
    \definecolor{outcolor}{rgb}{0.545, 0.0, 0.0}



    
    % Prevent overflowing lines due to hard-to-break entities
    \sloppy 
    % Setup hyperref package
    \hypersetup{
      breaklinks=true,  % so long urls are correctly broken across lines
      colorlinks=true,
      urlcolor=blue,
      linkcolor=darkorange,
      citecolor=darkgreen,
      }
    % Slightly bigger margins than the latex defaults
    
    \geometry{verbose,tmargin=1in,bmargin=1in,lmargin=1in,rmargin=1in}
    
    

    \begin{document}
    
    
    \maketitle
    
    

    

    \section{Introduction}


    The product offer of supermarkets is large and complex and it can be
very challenging for consumers to navigate into it. This complexity is
an obstacle for conducting basic actions such as: - comparing price
between supermarkets - find similar products in order to reach a
specific budget - switch products to match a specific behaviour
(organic, hallal, gluten-free\ldots{})

The objective of this project is to provide consumers with a navigation
tool that will his shopping easier more efficient and powerful.

The following steps have been followed: - Step 1: Crawl product
inventories from online supermarkets - Step 2: Data cleaning (avoid
redundancy, completion, formating) - Step 3: Build Distance Functions
between products to allow efficient substitutions - Step 4: Find the
best candidate for a given shopping cart


    \section{Crawling inventories}


    For this project, we have focused our efforts on 3 online supermakets:
Auchan, Monoprix and Simply. But the approach that we will detail below
can be easily applied to any supermarket website, as they all share a
similar architecture: - a welcome page with links to different
categories (beverage, meat, fruit\ldots{}) - inside each category, you
will have different level a sub-categories (alcohol, juice,
soda\ldots{}) - after a couple of categorization levels (usually between
1 and 5), you reach the product offer - by clicking on each product, you
reach a ``product page'' where detailed information (description,
energetic values, price\ldots{}) is displayed

Our crawlers are built in two steps: - navigate through the categories
to collect ``product page'' urls - extract relevant information for each
``product page''

The Python code for the first step is described below:

    \begin{Verbatim}[commandchars=\\\{\}]
{\color{incolor}In [{\color{incolor}}]:} \PY{k+kn}{import} \PY{n+nn}{requests}
       \PY{k+kn}{from} \PY{n+nn}{bs4} \PY{k+kn}{import} \PY{n}{BeautifulSoup}
       \PY{k+kn}{import} \PY{n+nn}{pandas} \PY{k+kn}{as} \PY{n+nn}{pd}
       \PY{k+kn}{import} \PY{n+nn}{re}
       
       \PY{c}{\PYZsh{}different levels of categorization}
       \PY{n}{rayonsA}\PY{o}{=}\PY{p}{[}\PY{p}{]}
       \PY{n}{rayonsB}\PY{o}{=}\PY{p}{[}\PY{p}{]}
       \PY{n}{rayonsC}\PY{o}{=}\PY{p}{[}\PY{p}{]}
       \PY{n}{rayonsD}\PY{o}{=}\PY{p}{[}\PY{p}{]}
       \PY{n}{rayonsE}\PY{o}{=}\PY{p}{[}\PY{p}{]}
       \PY{n}{productLink}\PY{o}{=}\PY{p}{[}\PY{p}{]}
       
       \PY{c}{\PYZsh{}start at welcome page and get first level categories}
       \PY{n}{url}\PY{o}{=}\PY{l+s}{\PYZsq{}}\PY{l+s}{http://www.livraison.simplymarket.fr/}\PY{l+s}{\PYZsq{}}
       \PY{n}{r} \PY{o}{=} \PY{n}{requests}\PY{o}{.}\PY{n}{get}\PY{p}{(}\PY{n}{url}\PY{p}{)}
       \PY{n}{soup} \PY{o}{=} \PY{n}{BeautifulSoup}\PY{p}{(}\PY{n}{r}\PY{o}{.}\PY{n}{text}\PY{p}{,}\PY{l+s}{\PYZsq{}}\PY{l+s}{html.parser}\PY{l+s}{\PYZsq{}}\PY{p}{)}
       \PY{n}{balises\PYZus{}a}\PY{o}{=}\PY{n}{soup}\PY{o}{.}\PY{n}{find\PYZus{}all}\PY{p}{(}\PY{l+s}{\PYZdq{}}\PY{l+s}{a}\PY{l+s}{\PYZdq{}}\PY{p}{,}\PY{n}{class\PYZus{}}\PY{o}{=}\PY{l+s}{\PYZsq{}}\PY{l+s}{linkMenu}\PY{l+s}{\PYZsq{}}\PY{p}{)}
       
       \PY{k}{for} \PY{n}{balise\PYZus{}a} \PY{o+ow}{in} \PY{n}{balises\PYZus{}a}\PY{p}{:}
       	\PY{n}{rayonsA}\PY{o}{.}\PY{n}{append}\PY{p}{(}\PY{l+s}{\PYZsq{}}\PY{l+s}{http://www.livraison.simplymarket.fr/}\PY{l+s}{\PYZsq{}}\PY{o}{+}\PY{n}{balise\PYZus{}a}\PY{o}{.}\PY{n}{get}\PY{p}{(}\PY{l+s}{\PYZsq{}}\PY{l+s}{href}\PY{l+s}{\PYZsq{}}\PY{p}{)}\PY{p}{)}
       
       \PY{c}{\PYZsh{}navigate in each Level\PYZhy{}1 category}
       \PY{k}{for} \PY{n}{rayonA} \PY{o+ow}{in} \PY{n}{rayonsA}\PY{p}{:}
       	\PY{n}{r} \PY{o}{=} \PY{n}{requests}\PY{o}{.}\PY{n}{get}\PY{p}{(}\PY{n}{rayonA}\PY{p}{)}
       	\PY{n}{soup} \PY{o}{=} \PY{n}{BeautifulSoup}\PY{p}{(}\PY{n}{r}\PY{o}{.}\PY{n}{text}\PY{p}{,}\PY{l+s}{\PYZsq{}}\PY{l+s}{html.parser}\PY{l+s}{\PYZsq{}}\PY{p}{)}
       	\PY{n}{balises\PYZus{}a}\PY{o}{=}\PY{n}{soup}\PY{o}{.}\PY{n}{find\PYZus{}all}\PY{p}{(}\PY{l+s}{\PYZdq{}}\PY{l+s}{a}\PY{l+s}{\PYZdq{}}\PY{p}{,}\PY{n}{class\PYZus{}}\PY{o}{=}\PY{l+s}{\PYZsq{}}\PY{l+s}{lienProduit}\PY{l+s}{\PYZsq{}}\PY{p}{)}
       	\PY{n}{balises\PYZus{}tr}\PY{o}{=}\PY{n}{soup}\PY{o}{.}\PY{n}{find\PYZus{}all}\PY{p}{(}\PY{l+s}{\PYZdq{}}\PY{l+s}{tr}\PY{l+s}{\PYZdq{}}\PY{p}{,}\PY{n}{class\PYZus{}}\PY{o}{=}\PY{l+s}{\PYZsq{}}\PY{l+s}{trLibelle}\PY{l+s}{\PYZsq{}}\PY{p}{)}
       
       \PY{c}{\PYZsh{}find product pages, if any, and add it to the list}
       	\PY{k}{for} \PY{n}{balise\PYZus{}a} \PY{o+ow}{in} \PY{n}{balises\PYZus{}a}\PY{p}{:}
       		\PY{n}{productLink}\PY{o}{.}\PY{n}{append}\PY{p}{(}\PY{l+s}{\PYZsq{}}\PY{l+s}{http://www.livraison.simplymarket.fr/}\PY{l+s}{\PYZsq{}}\PY{o}{+}\PY{n}{balise\PYZus{}a}\PY{o}{.}\PY{n}{get}\PY{p}{(}\PY{l+s}{\PYZsq{}}\PY{l+s}{href}\PY{l+s}{\PYZsq{}}\PY{p}{)}\PY{p}{)}
       
       \PY{c}{\PYZsh{}if no product pages can be found, find Level\PYZhy{}2 categories}
       	\PY{k}{if} \PY{n}{balises\PYZus{}a}\PY{o}{==}\PY{p}{[}\PY{p}{]}\PY{p}{:}
       		\PY{k}{for} \PY{n}{balise\PYZus{}tr} \PY{o+ow}{in} \PY{n}{balises\PYZus{}tr}\PY{p}{:}
       			\PY{n}{rayonsB}\PY{o}{.}\PY{n}{append}\PY{p}{(}\PY{l+s}{\PYZsq{}}\PY{l+s}{http://www.livraison.simplymarket.fr/}\PY{l+s}{\PYZsq{}}\PY{o}{+}\PY{n}{balise\PYZus{}tr}\PY{o}{.}\PY{n}{find}\PY{p}{(}\PY{l+s}{\PYZdq{}}\PY{l+s}{a}\PY{l+s}{\PYZdq{}}\PY{p}{)}\PY{o}{.}\PY{n}{get}\PY{p}{(}\PY{l+s}{\PYZsq{}}\PY{l+s}{href}\PY{l+s}{\PYZsq{}}\PY{p}{)}\PY{p}{)}
\end{Verbatim}

    This algorithm is repeated over each level of categorization until no
new category is discovered. At the end, the variable productLink is a
list that contains all the ``product page'' urls and we can move to the
second step.

The second step consist in scrapping the relevant part of the html code
of each ``product page'' that contains the following fields: 
\begin{itemize}
\item product\_name
\item brand 
\item quantity (eg 6 for 6x33cL) 
\item weight/volume (eg 33 for 6x33cL) 
\item weight/volume\_total (eg 200 for 6x33cL) 
\item unit (eg cL)
\item description 
\item ingredients 
\item preservation 
\item nutritional details
\item origin 
\item price 
\item price per unit
\end{itemize}


    \section{Data Cleaning}


    Once the html code containing these data has been scrapped, it needs to
be parsed and cleaned in order to fill the table with the correct
values. Ideally, we would have stored the code in a text database, such
as MongoDB, and then build a parser to transfer information from MongDB
to a standard relationnal database. For this project, we have scrapped
and clean the data at the same time and build a csv file gathering the
data.

First, the parsing/cleaning stage consists in understanding the
structure of the html code. For example:

    \begin{Verbatim}[commandchars=\\\{\}]
{\color{incolor}In [{\color{incolor}}]:}     
           \PY{o}{\PYZlt{}}\PY{n}{div} \PY{n}{class}\PY{o}{=}\PY{l+s}{\PYZdq{}}\PY{l+s}{redactionnel}\PY{l+s}{\PYZdq{}}\PY{o}{\PYZgt{}}
               \PY{o}{\PYZlt{}}\PY{n}{div} \PY{n}{class}\PY{o}{=}\PY{l+s}{\PYZdq{}}\PY{l+s}{texteProduit}\PY{l+s}{\PYZdq{}}\PY{o}{\PYZgt{}}
                   \PY{o}{\PYZlt{}}\PY{n}{h1}\PY{o}{\PYZgt{}}\PY{n}{FOIE} \PY{n}{GRAS} \PY{n}{DE} \PY{n}{CANARD} \PY{n}{ENTIER} \PY{n}{SUD} \PY{n}{OUEST} \PY{n}{GASTRONOMIQUE} \PY{n}{MONTFORT} \PY{l+m+mi}{300}\PY{n}{G} \PY{o}{\PYZhy{}} \PY{n}{Montfort}\PY{o}{\PYZlt{}}\PY{o}{/}\PY{n}{h1}\PY{o}{\PYZgt{}}
               \PY{o}{\PYZlt{}}\PY{o}{/}\PY{n}{div}\PY{o}{\PYZgt{}}
                       \PY{o}{\PYZlt{}}\PY{n}{div} \PY{n}{class}\PY{o}{=}\PY{l+s}{\PYZdq{}}\PY{l+s}{texteProduit}\PY{l+s}{\PYZdq{}}\PY{o}{\PYZgt{}}
                   \PY{o}{\PYZlt{}}\PY{n}{label}\PY{o}{\PYZgt{}}\PY{n}{Prix} \PY{n}{quantit}\PY{err}{é}\PY{o}{\PYZlt{}}\PY{o}{/}\PY{n}{label}\PY{o}{\PYZgt{}}\PY{o}{\PYZlt{}}\PY{n}{br}\PY{o}{\PYZgt{}}
                   \PY{l+m+mi}{99}\PY{p}{,}\PY{l+m+mi}{67}\PY{o}{\PYZam{}}\PY{n}{nbsp}\PY{p}{;}\PY{n}{eur}\PY{o}{/}\PY{n}{Kg}
               \PY{o}{\PYZlt{}}\PY{o}{/}\PY{n}{div}\PY{o}{\PYZgt{}}
               
                       \PY{o}{\PYZlt{}}\PY{n}{div} \PY{n}{class}\PY{o}{=}\PY{l+s}{\PYZdq{}}\PY{l+s}{texteProduit}\PY{l+s}{\PYZdq{}}\PY{o}{\PYZgt{}}
                   \PY{o}{\PYZlt{}}\PY{n}{label}\PY{o}{\PYZgt{}}\PY{n}{Descriptif}\PY{o}{\PYZlt{}}\PY{o}{/}\PY{n}{label}\PY{o}{\PYZgt{}}\PY{o}{\PYZlt{}}\PY{n}{br}\PY{o}{\PYZgt{}}
                   \PY{n}{Foie} \PY{n}{gras} \PY{n}{de} \PY{n}{canard} \PY{n}{entier} \PY{n}{du} \PY{n}{Sud} \PY{n}{Ouest}\PY{o}{.}
               \PY{o}{\PYZlt{}}\PY{o}{/}\PY{n}{div}\PY{o}{\PYZgt{}}
               
                       \PY{o}{\PYZlt{}}\PY{n}{div} \PY{n}{class}\PY{o}{=}\PY{l+s}{\PYZdq{}}\PY{l+s}{texteProduit}\PY{l+s}{\PYZdq{}}\PY{o}{\PYZgt{}}
                   \PY{o}{\PYZlt{}}\PY{n}{label}\PY{o}{\PYZgt{}}\PY{n}{Avantages}\PY{o}{\PYZlt{}}\PY{o}{/}\PY{n}{label}\PY{o}{\PYZgt{}}\PY{o}{\PYZlt{}}\PY{n}{br}\PY{o}{\PYZgt{}}
                   \PY{o}{\PYZlt{}}\PY{n}{div}\PY{o}{\PYZgt{}}\PY{n}{La} \PY{n}{recette} \PY{n}{du} \PY{n}{Gastronomique} \PY{n}{est} \PY{n}{inspir}\PY{err}{é}\PY{n}{e} \PY{n}{des} \PY{n}{pr}\PY{err}{é}\PY{n}{parations} \PY{n}{des} \PY{n}{plus} \PY{n}{grands} \PY{n}{Chefs} \PY{p}{:} \PY{n}{une} \PY{n}{cuisson} \PY{n}{lente} \PY{n}{et} \PY{err}{à} \PY{n}{basse} \PY{n}{temp}\PY{err}{é}\PY{n}{rature} \PY{n}{qui} \PY{n}{permet} \PY{n}{de} \PY{n}{pr}\PY{err}{é}\PY{n}{server} \PY{n}{toutes} \PY{n}{les} \PY{n}{saveurs} \PY{n}{du} \PY{n}{foie} \PY{n}{gras} \PY{n}{et} \PY{n}{de} \PY{n}{sublimer} \PY{n}{sa} \PY{n}{texture}\PY{p}{,} \PY{n}{comme} \PY{n}{dans} \PY{n}{les} \PY{n}{grands} \PY{n}{restaurants}\PY{o}{.}\PY{o}{\PYZlt{}}\PY{o}{/}\PY{n}{div}\PY{o}{\PYZgt{}}\PY{o}{\PYZlt{}}\PY{n}{div}\PY{o}{\PYZgt{}}\PY{o}{\PYZam{}}\PY{n}{nbsp}\PY{p}{;}\PY{o}{\PYZhy{}} \PY{n}{Un} \PY{n}{label} \PY{n}{IGP} \PY{n}{Sud}\PY{o}{\PYZhy{}}\PY{n}{Ouest} \PY{n}{gage} \PY{n}{de} \PY{n}{qualit}\PY{err}{é} \PY{n}{et} \PY{n}{de} \PY{n}{tra}\PY{err}{ç}\PY{n}{abilit}\PY{err}{é}\PY{o}{.}\PY{o}{\PYZlt{}}\PY{o}{/}\PY{n}{div}\PY{o}{\PYZgt{}}\PY{o}{\PYZlt{}}\PY{n}{div}\PY{o}{\PYZgt{}}\PY{o}{\PYZhy{}} \PY{n}{Un} \PY{n}{Foie} \PY{n}{Gras} \PY{n}{Entier} \PY{p}{:} \PY{n}{le} \PY{n}{meilleur} \PY{n}{du} \PY{n}{Foie} \PY{n}{Gras}\PY{o}{.}\PY{o}{\PYZlt{}}\PY{o}{/}\PY{n}{div}\PY{o}{\PYZgt{}}\PY{o}{\PYZlt{}}\PY{n}{div}\PY{o}{\PYZgt{}}\PY{o}{\PYZhy{}} \PY{n}{un} \PY{n}{Foie} \PY{n}{Gras} \PY{n}{d}\PY{err}{é}\PY{n}{moulable} \PY{n}{pour} \PY{n}{des} \PY{n}{tranches} \PY{n}{trap}\PY{err}{è}\PY{n}{zes} \PY{n}{r}\PY{err}{é}\PY{n}{guli}\PY{err}{è}\PY{n}{res}\PY{o}{\PYZlt{}}\PY{o}{/}\PY{n}{div}\PY{o}{\PYZgt{}}\PY{o}{\PYZlt{}}\PY{n}{div}\PY{o}{\PYZgt{}}\PY{o}{\PYZhy{}} \PY{n}{pour} \PY{l+m+mi}{8} \PY{err}{à} \PY{l+m+mi}{9} \PY{n}{parts}\PY{o}{.}\PY{o}{\PYZlt{}}\PY{o}{/}\PY{n}{div}\PY{o}{\PYZgt{}}
               \PY{o}{\PYZlt{}}\PY{o}{/}\PY{n}{div}\PY{o}{\PYZgt{}}
               
                       \PY{o}{\PYZlt{}}\PY{n}{div} \PY{n}{class}\PY{o}{=}\PY{l+s}{\PYZdq{}}\PY{l+s}{texteProduit}\PY{l+s}{\PYZdq{}}\PY{o}{\PYZgt{}}
                   \PY{o}{\PYZlt{}}\PY{n}{label}\PY{o}{\PYZgt{}}\PY{n}{Ingr}\PY{err}{é}\PY{n}{dients}\PY{o}{\PYZlt{}}\PY{o}{/}\PY{n}{label}\PY{o}{\PYZgt{}}\PY{o}{\PYZlt{}}\PY{n}{br}\PY{o}{\PYZgt{}}
                   \PY{n}{Foie} \PY{n}{gras} \PY{n}{de} \PY{n}{canard}\PY{p}{,} \PY{n}{sel}\PY{p}{,} \PY{n}{Armagnac}\PY{p}{,} \PY{n}{Porto}\PY{p}{,} \PY{n}{poivre}\PY{p}{,} \PY{n}{sucre}\PY{p}{,} \PY{n}{antioxydant} \PY{p}{:} \PY{n}{ascorbate} \PY{n}{de} \PY{n}{sodium}\PY{p}{,} \PY{n}{conservateur} \PY{p}{:} \PY{n}{nitrite} \PY{n}{de} \PY{n}{sodium}\PY{o}{.}
               \PY{o}{\PYZlt{}}\PY{o}{/}\PY{n}{div}\PY{o}{\PYZgt{}}
               
                       \PY{o}{\PYZlt{}}\PY{n}{div} \PY{n}{class}\PY{o}{=}\PY{l+s}{\PYZdq{}}\PY{l+s}{texteProduit}\PY{l+s}{\PYZdq{}}\PY{o}{\PYZgt{}}
                   \PY{o}{\PYZlt{}}\PY{n}{label}\PY{o}{\PYZgt{}}\PY{n}{Conservation}\PY{o}{\PYZlt{}}\PY{o}{/}\PY{n}{label}\PY{o}{\PYZgt{}}\PY{o}{\PYZlt{}}\PY{n}{br}\PY{o}{\PYZgt{}}
                   \PY{n}{A} \PY{n}{conserver} \PY{n}{au} \PY{n}{r}\PY{err}{é}\PY{n}{frig}\PY{err}{é}\PY{n}{rateur} \PY{n}{entre} \PY{l+m+mi}{0}\PY{err}{°}\PY{n}{C} \PY{n}{et} \PY{o}{+}\PY{l+m+mi}{4}\PY{err}{°}\PY{n}{C}\PY{o}{.} \PY{n}{A} \PY{n}{consommer} \PY{n}{rapidement} \PY{n}{apr}\PY{err}{è}\PY{n}{s} \PY{n}{ouverture}\PY{o}{.}
               \PY{o}{\PYZlt{}}\PY{o}{/}\PY{n}{div}\PY{o}{\PYZgt{}}
               
                       \PY{o}{\PYZlt{}}\PY{n}{div} \PY{n}{class}\PY{o}{=}\PY{l+s}{\PYZdq{}}\PY{l+s}{texteProduit}\PY{l+s}{\PYZdq{}}\PY{o}{\PYZgt{}}
                   \PY{o}{\PYZlt{}}\PY{n}{label}\PY{o}{\PYZgt{}}\PY{n}{Renseignements} \PY{n}{pratiques}\PY{o}{\PYZlt{}}\PY{o}{/}\PY{n}{label}\PY{o}{\PYZgt{}}\PY{o}{\PYZlt{}}\PY{n}{br}\PY{o}{\PYZgt{}}
                   \PY{o}{\PYZlt{}}\PY{n}{p}\PY{o}{\PYZgt{}}\PY{n}{Sortir} \PY{n}{le} \PY{n}{produit} \PY{n}{du} \PY{n}{r}\PY{err}{é}\PY{n}{frig}\PY{err}{é}\PY{n}{rateur} \PY{n}{quelques} \PY{n}{minutes} \PY{n}{avant} \PY{n}{de} \PY{n}{le} \PY{n}{d}\PY{err}{é}\PY{n}{guster}\PY{o}{.}\PY{o}{\PYZlt{}}\PY{o}{/}\PY{n}{p}\PY{o}{\PYZgt{}}\PY{o}{\PYZlt{}}\PY{n}{p}\PY{o}{\PYZgt{}}\PY{n}{Euralis} \PY{n}{Gastronomie}\PY{o}{\PYZlt{}}\PY{n}{br}\PY{o}{\PYZgt{}}\PY{n}{Av}\PY{o}{.} \PY{n}{Gaston} \PY{n}{Phoebus} \PY{l+m+mi}{64230} \PY{n}{LESCAR}\PY{o}{\PYZam{}}\PY{n}{nbsp}\PY{p}{;}\PY{o}{\PYZlt{}}\PY{o}{/}\PY{n}{p}\PY{o}{\PYZgt{}}
               \PY{o}{\PYZlt{}}\PY{o}{/}\PY{n}{div}\PY{o}{\PYZgt{}}
               
                       \PY{o}{\PYZlt{}}\PY{n}{div} \PY{n}{class}\PY{o}{=}\PY{l+s}{\PYZdq{}}\PY{l+s}{texteProduit}\PY{l+s}{\PYZdq{}}\PY{o}{\PYZgt{}}
               
               \PY{o}{\PYZlt{}}\PY{n}{table} \PY{n}{width}\PY{o}{=}\PY{l+s}{\PYZdq{}}\PY{l+s}{100}\PY{l+s}{\PYZpc{}}\PY{l+s}{\PYZdq{}} \PY{n}{cellspacing}\PY{o}{=}\PY{l+s}{\PYZdq{}}\PY{l+s}{0}\PY{l+s}{\PYZdq{}} \PY{n}{cellpadding}\PY{o}{=}\PY{l+s}{\PYZdq{}}\PY{l+s}{0}\PY{l+s}{\PYZdq{}}\PY{o}{\PYZgt{}}
                   \PY{o}{\PYZlt{}}\PY{n}{tbody}\PY{o}{\PYZgt{}}\PY{o}{\PYZlt{}}\PY{n}{tr}\PY{o}{\PYZgt{}}
\end{Verbatim}

    For this example, we can observe that each piece of information is
preceded by its label (eg `Conservation'). The parser will try to find
these labels and if they exist, it will take the information that comes
right after and store it in the relevant variable.

    \begin{Verbatim}[commandchars=\\\{\}]
{\color{incolor}In [{\color{incolor}}]:} \PY{k}{if} \PY{l+s}{\PYZsq{}}\PY{l+s}{Conservation}\PY{l+s}{\PYZsq{}} \PY{o+ow}{in} \PY{n}{reduced\PYZus{}prod\PYZus{}info}\PY{p}{:}
       			\PY{n}{index\PYZus{}conservation}\PY{o}{=}\PY{n}{reduced\PYZus{}prod\PYZus{}info}\PY{o}{.}\PY{n}{index}\PY{p}{(}\PY{l+s}{\PYZsq{}}\PY{l+s}{Conservation}\PY{l+s}{\PYZsq{}}\PY{p}{)}
       			\PY{n}{conservation}\PY{o}{=}\PY{n}{reduced\PYZus{}prod\PYZus{}info}\PY{p}{[}\PY{n}{index\PYZus{}conservation}\PY{o}{+}\PY{l+m+mi}{1}\PY{p}{]}
\end{Verbatim}

    Some data are stored in a different format than expected in our
database. To solve this, we have used regular expressions in order to
recognize patterns and extract exactly what we were looking for.
Example: FOIE GRAS DE CANARD ENTIER SUD OUEST GASTRONOMIQUE MONTFORT
300G - Montfort

We want to have: name = FOIE GRAS DE CANARD ENTIER SUD OUEST
GASTRONOMIQUE MONTFORT quantity = 1 weight/volume = 300 unit=G

In Python, this will translate:

    \begin{Verbatim}[commandchars=\\\{\}]
{\color{incolor}In [{\color{incolor}}]:} \PY{n}{title}\PY{o}{=}\PY{n}{re}\PY{o}{.}\PY{n}{match}\PY{p}{(}\PY{l+s}{r\PYZsq{}}\PY{l+s}{(.*)}\PY{l+s}{\PYZbs{}}\PY{l+s}{s(}\PY{l+s}{\PYZbs{}}\PY{l+s}{d+}\PY{l+s}{\PYZbs{}}\PY{l+s}{.?}\PY{l+s}{\PYZbs{}}\PY{l+s}{d+?)(KG|G|ML|CL|L)}\PY{l+s}{\PYZsq{}}\PY{p}{,}\PY{n}{nom}\PY{p}{)}
       		\PY{k}{if} \PY{n}{title}\PY{o}{!=}\PY{n+nb+bp}{None}\PY{p}{:}
       			\PY{n}{nom}\PY{o}{=}\PY{n}{title}\PY{o}{.}\PY{n}{group}\PY{p}{(}\PY{l+m+mi}{1}\PY{p}{)}
       			\PY{n}{quantite}\PY{o}{=}\PY{l+m+mi}{1}
       			\PY{n}{poids\PYZus{}volume\PYZus{}total}\PY{o}{=}\PY{n}{title}\PY{o}{.}\PY{n}{group}\PY{p}{(}\PY{l+m+mi}{2}\PY{p}{)}
       			\PY{n}{unite}\PY{o}{=}\PY{n}{title}\PY{o}{.}\PY{n}{group}\PY{p}{(}\PY{l+m+mi}{3}\PY{p}{)}
\end{Verbatim}

    By iterating this operation on most observed formats, we managed to
build a clean dataset available for analysis.


    \section{Distance between products}


    \begin{Verbatim}[commandchars=\\\{\}]
{\color{incolor}In [{\color{incolor}}]:} 
\end{Verbatim}


    % Add a bibliography block to the postdoc
    
    
    
    \end{document}
